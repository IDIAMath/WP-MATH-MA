% ..............................................................................
% Demo of the fau-beamer template.
%
% Copyright 2022 by Tim Roith <tim.roith@fau.de>
%
% This program can be redistributed and/or modified under the terms
% of the GNU Public License, version 2.
%
% ------------------------------------------------------------------------------
\documentclass[final]{beamer}


% ========================================================================================
% Theme: inner, outer, font and colors
% ----------------------------------------------------------------------------------------
\usepackage[institute=Nat,
			ExtraLogo = template-art/DDS_blau.png,
			%WordMark=None
			aspectratio=169,
			size=18
		   ]{styles/beamerthemefau}
% ----------------------------------------------------------------------------------------
% Input and output encoding
\usepackage[T1]{fontenc}
\usepackage[utf8]{inputenc}
% ----------------------------------------------------------------------------------------
% Language settings
\usepackage[english]{babel}


% ========================================================================================
% Fonts
% - Helvet is loaded by styles/beamerfonts
% - We use serif for math environements
% - isomath is used for upGreek letters
% ----------------------------------------------------------------------------------------
\usepackage{isomath}
\usefonttheme[onlymath]{serif}
\usepackage{exscale}
\usepackage{anyfontsize}
\setbeamercolor{alerted text}{fg=BaseColor}
% ----------------------------------------------------------------------------------------
% custom commands for symbols
\usepackage{styles/symbols}


% ========================================================================================
% Setup for Titlepage
% ----------------------------------------------------------------------------------------
\title[Multivariate Analysis]{Interactive Formative Assessments in Multivariate Analysis}
\subtitle{Now possibilities in JSXGraph}
\author[Ammon, Rathmann]{Nico Ammon, Wigand Rathmann}
\institute[FAU]{Friedrich-Alexander Universit\"{a}t Erlangen-N\"{u}rnberg, Department of Data Science}
\date{1. Juni 2023}


% ========================================================================================
% Bibliography
% ----------------------------------------------------------------------------------------
\usepackage{csquotes}
\usepackage[style=alphabetic, %alternatively: numeric, numeric-comp, and other from biblatex
			defernumbers=true,
			useprefix=true,%
			giveninits=true,%
			hyperref=true,%
			autocite=inline,%
			maxcitenames=5,%
			maxbibnames=20,%
			uniquename=init,%
			sortcites=true,% sort citations when multiple entries are passed to one cite command
			doi=true,%
			isbn=false,%
			url=false,%
			eprint=false,%
			backend=biber%
		   ]{biblatex}
\addbibresource{bibliography.bib}
\setbeamertemplate{bibliography item}[text]


% ========================================================================================
% Hyperref and setup
% ----------------------------------------------------------------------------------------
\usepackage{hyperref}
\hypersetup{
	colorlinks = true,
	final=true,
	plainpages=false,
	pdfstartview=FitV,
	pdftoolbar=true,
	pdfmenubar=true,
	pdfencoding=auto,
	psdextra,
	bookmarksopen=true,
	bookmarksnumbered=true,
	breaklinks=true,
	linktocpage=true,
	urlcolor=BaseColor,
	citecolor=BaseColor,
	linkcolor=BaseColor
}


% ========================================================================================
% Additional packages
% ----------------------------------------------------------------------------------------



% ========================================================================================
% Various custom commands
% ----------------------------------------------------------------------------------------
\pdfsuppresswarningpagegroup=1 %solves the PDF inclusion problem
% Change color for cite locally
\newcommand{\colorcite}[3]{{\hypersetup{citecolor=#1}{\cite[#2]{#3}}}}
% ----------------------------------------------------------------------------------------


% ========================================================================================
% The main document
% ----------------------------------------------------------------------------------------
\begin{document}
% Title page
\begin{trueplainframe}{}
\titlepage%
\end{trueplainframe}

% Introduction
\begin{frame}[t]{Introduction}{What will be demonstrated?}

\begin{block}{Focus}
\begin{itemize}
\item Visualization aspects 
    \begin{itemize}
    \item coordinate transformation:

    \begin{itemize}
    \item    polar coordinates,
    \item    spherical coordinates,
    \item    general coordinate transformation,
    \item    cylindriacal coordinate transformation
    \end{itemize}
    \item calculus of function with two variable,
    \item visualization of vector fields.
    \end{itemize}
    \end{itemize}
\end{block}
\vfill

\centerline{\textbf{ERASMUS+ Interactive Digital Assessment in Mathematics}}

\vfill

    \end{frame}

% Outline
%\begin{frame}[t]{Outline}
%\tableofcontents
%\end{frame}


% input exmple sections

\section{Integration 2D}
\begin{frame}[t]{Coordinate Transformation}

\begin{block}{Polar Coordinates}
 \(T:[0,\infty)\times[0,2\pi)\to\mathbb{R}^2\) with
 \(\begin{pmatrix}r\\ \phi \end{pmatrix}\mapsto  \begin{pmatrix}r\cos(\phi)\\r\sin(\phi)\end{pmatrix}\)
\end{block}
\end{frame}

%\begin{frame}[t]
%
%
%\begin{center}
%\includegraphics[width=0.45\textwidth]{01PolarCoord.png}
%\end{center}
%\end{frame}
%
%\begin{frame} [t]{Curvilinear Bounded Domains}
%
%\begin{block}{Integration areas}
%\begin{itemize}
%\item projectable areas
% \[M_y:=\{(x,y)^\top\in \mathbb{R}^2\,: \, a\leqslant x \leqslant b,\; \underline{y}(x)\leqslant y \leqslant \bar{y}_u(x)\}\]
% \[M_x:=\{(x,y)^\top\in \mathbb{R}^2\,: \, c\leqslant y \leqslant d,\; \underline{x}(y)\leqslant x \leqslant \bar{x}_u(y)\}\]
%\item \textbf{Idea:} Identify curves of an arbitrary bounded set?
%\end{itemize}
%\end{block}
%
%\hfill\includegraphics[width=0.1\textwidth]{02Input.png}\hfill\includegraphics[width=0.3\textwidth]{03Area.png}\hfill\mbox{}
%\end{frame}

\section{Integration 3D}
\begin{frame} [t]{Spherical Coordinates}
\begin{block}{Transformation}
Transformation
        \(T:\mathbb{R}^+_0\times [0,2\pi)\times [0,\pi]\to\mathbb{R}^3\) with
        \[\begin{pmatrix}
            r\\ \phi \\ \psi\end{pmatrix}\mapsto  
            \begin{pmatrix}r\cos(\phi)\sin(\psi)\\ 
            r\sin(\phi)\sin(\psi)\\ \r\cos(\psi)\end{pmatrix}\]
\end{block}
%\begin{center}
%\includegraphics[width=0.4\textwidth]{04sphere.png}
%\end{center}
\end{frame}

\begin{frame}[t]{Coordinate Transformation}
\begin{block}{General setting}
Given box \(H=[u_1,u_2]\times[v_1,v_2]\times[w_1,w_2]\) and a transformation
        \(T:\mathbb{R}^3\to\mathbb{R}^3\) with
        \[\begin{pmatrix}u\\v\\w\end{pmatrix}\mapsto  \begin{pmatrix}T_1(u,v,w)\\T_2(u,v,w)\\T_3(u,v,w)\end{pmatrix}\]

        To display the transformed set \( M=T([u_1,u_2]\times[v_1,v_2]\times[w_1,w_2])\) one needs six surfaces:
        \begin{align*}S_1(v,w) & = T(u_1,v,w) & S_3(u,w) & = T(u,v_1,w) & S_5(u,v) & = T(u,v,w_1)  \\
                        S_2(v,w) & = T(u_2,v,w)& S_4(u,w) & = T(u,v_2,w)& S_6(u,v) & = T(u,v,w_2)
                        \end{align*}
\end{block}


\begin{example}{3D Transform}
          \begin{align*} T_1(u,v,w)& = u \cos(v) \sin(w) \\
        T_2(u,v,w)& = u \cos(v) \sin(w)  \\
        T_3(u,v,w)& = u\cos{w}
        \end{align*}

\end{example}
\end{frame}

\begin{frame}[t]{Teaching ans Learning aspects}
\end{frame}

\begin{frame} [t]{Challenges for other users}

\end{frame}
\section{Calculus of functions with two variables}
\begin{frame} [t]{}
\begin{block}{Example function}
\[f:\R^2\to\R^2, \mathbf{x}\mapsto  a_1  \cos(a_2\pi x_1 )\cos( a_3x_2)\]
\end{block}

\begin{block}{Things to consider}
\begin{itemize}
\item substitute sliders in JSXGraph applet by random values
\item transfer function from question variables to JSXGraph
\centerline{\texttt{F:a1 * cos(\%pi*a2 * x )* cos( a3 * y);}}
\item attach local derivative at the graph
\end{itemize}
\end{block}
\end{frame}
\begin{frame} [t]{Tangential plane}
\begin{center}
\includegraphics[width=0.45\textwidth]{05Tangentplane.png}
\end{center}
\end{frame}

\begin{frame} [t]{Local Quadratic Approximation}
\begin{minipage}[t]{12cm}
\begin{center}
\includegraphics[width=\textwidth]{06Taylor.png}
\end{center}
\end{minipage}\hfill
\begin{minipage}[b]{17cm}
\texttt{%
// Maxima\\
F:a1 * cos(\%pi*a2 * x )* cos( a3 * y);\\
Fdx:diff(F,x);\\
Fdy:diff(F,y);\\
Fdxx:diff(Fdx,x);\\
Fdxy:diff(Fdy,x);\\
Fdyy:diff(Fdy,y);\\
%
// JSXGraph, JessieCode\\
var F =  board.jc.snippet('{\#F\#}', true, 'x,y');\\
var Fdx =  board.jc.snippet('{\#Fdx\#}', true, 'x,y');\\
var Fdx =  board.jc.snippet('{\#Fdy\#}', true, 'x,y');\\
var Fdxx =  board.jc.snippet('{\#Fdxx\#}', true, 'x,y');\\
var Fdxy =  board.jc.snippet('{\#Fdxy\#}', true, 'x,y');\\
var Fdyy =  board.jc.snippet('{\#Fdyy\#}', true, 'x,y');\\
}
\end{minipage}
\end{frame}

\begin{frame} [t]{Fun Facts}
\begin{block}{Transfer \(\pi\) from STACK to JSXGraph}
\begin{tabular}{ll}
Maxima-Notation &\texttt{\%pi}\\
JessieCode-Notation& \texttt{PI}
\end{tabular}		

\end{block}

\begin{example}[Code]
\texttt{var txtraw = '{\#F\#}';\\
		                               txtraw=txtraw.replace(/\%pi/g, \grqq{}PI\grqq{});\\
		                               var F =  board.jc.snippet(txraw, true, 'x,y');
}
\end{example}

\begin{block}{Transfer 3D Coordinates from JSXGraph to STACK }
\begin{itemize}
  \item point in domain is a 3D object in JSXGraph
  \item bind to STACK at time for 2D points implemented
\end{itemize}
\end{block}
\begin{example}[Transfer coordinates from 3D to 2D]
\texttt{%
var p1 =board.create('point', [function () \{return A.D3.X();\} ,\\
\phantom{var p1 =board.create('point', [}function () \{return A.D3.Y();\}],\{visible:false\});\\
		stack\_jxg.bind\_point(ans1Ref,p1);
}
\end{example}


\end{frame}

\section{Vector Fields}

\begin{frame} [t]{Draw vector fields}

\begin{minipage}[b]{12cm}
\begin{example}[Vector field in a box]
Given           \begin{align*} V_1(x,y,z)& = -v \\
        V_2(x,y,z)& = u  \\
        V_3(x,y,z)& = 0
        \end{align*}
        \vspace*{15ex}
\end{example}
\end{minipage}\hfill
\begin{minipage}[t]{12cm}
\includegraphics[width=\textwidth]{07Vectorfield.png}
\end{minipage}\hfill\mbox{}
\end{frame} 

\begin{frame} [t]{}

\end{frame}
\end{document} 