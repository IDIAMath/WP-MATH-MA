\documentclass[a4paper,12pt]{article}
\usepackage{amsmath}
\usepackage{amssymb}
\usepackage{times}
\usepackage{graphicx}
\renewcommand{\familydefault}{\sfdefault}

\begin{document}
\begin{abstract}
\begin{center}
{\large {\bf Interactive Formative Assessments in Multivariate Analysis}} \\[+.1in]
{\bf Nico Ammon, Wigand Rathmann\\[1ex]
Friedrich-Alexander-Universit\"{a}t Erlangen-N\"{u}rnberg (FAU)\\
Faculty of Sciences \\
Cauerstra{\ss}e 11\\
91058 Erlangen
} \\ [+.2in]
\end{center}


\noindent
Multivariate analysis (MA) is a mathematical field present in virtually all engineering studies. While a strictly formal description of problems of MA is often not initially intuitive to students, many problems can be graphically presented in an intelligible way. One part of the Erasmus+ project IDIAM is to develop a set of problems of MA, which can be used by teachers for formative assessment of MA and by students to tackle the topic in a more intuitive way.
We chosen to start with examples from coordinate transformation, calculus of function with two variable and visualization of vector fields.

As the development of such problems undergoes many steps, there are currently problems in a wide range of completeness, ranging from ideas of tasks to essentially complete problems with a detailed documentation that only lack some sort of convention in the style of programming. Since the goal is to easily enable slight modifications, both a readily available documentation together with the didactical objectives as well as a generally uniform structure of programming are vital to the success of the project.

In the presentation, some problems are demonstrated. We will show the usage of randomization of STACK and the linkage to the 3D JSXGraph diagrams and the interaction between the diagram and the students answers. A brief description of the graphical representation using JSXGraph is given as well. The concept of specific feedback using potential response trees based on the student answers is elaborated.

Additionally, some of the challenges arising in the development of the tasks are presented. We hope to be able to give some helpful advice to other groups working on similar projects.
\end{abstract}
\end{document}

